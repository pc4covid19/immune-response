\documentclass[12point]{article}

\usepackage{hyperref}

\begin{document}

\section{v2 prototype}

\subsection{Transport}
We model external virus as a concentration that diffuses and can be uptake or exported. We use the standard PhysiCell equations: 

\begin{eqnarray}
\frac{\partial \rho }{\partial t} & = & 
D \nabla^2 \rho - \lambda \rho + 
\sum_{\textrm{cells }i} 
\delta( \vec{x} - \vec{x}_i ) \left[ 
-V_i U_i \rho + E_i 
\right]
\end{eqnarray}
where $D$ is the diffusion coefficient, $\lambda$ is the decay and/or removal rate, 
$U$ is the cell's uptake rate (see more below), and $E$ is the net export rate (see more below). 

We use a PDE like this for viral particles $V$, uncoated virus $U$ (which might be released by dead cells), viral RNA $R$ (might be released by dead cells), viral protein $P$ (might be released by dead cells), and also 
release assembled virus (used as a technical assist on viral internalization and export dynamics--all virions in this last equation are immediately transferred to the viral concentration. See below). 

\subsection{Internal virus model}
Let's define the following variables: \\

\begin{centering}
\begin{tabular}{rl} 
$V$ & virions that have been released from internalized ACE2 receptors in the cell (more below). \\
$U$ & viruses that have been uncoated \\
$R$ & RNA that has been delivered from uncoated virus $U$ \\
$P$ & (full sets) of viral proteins that have been synthesized from RNA $R$ \\
$A$ & assembled virions that are ready for export. 
\end{tabular}
\end{centering}
% Let's model just the internal processes, noting that PhysiCell automatically takes care of virus intake (source for $V$) and export (sink for $A$). 

Our placeholder model of this is a system of ODEs: 
\begin{eqnarray}
\frac{dV}{dt} & = & -r_\textrm{uncoat} V  + \overbrace{ \textrm{Source}_V }^\textrm{from receptor submodel} \\
\frac{dU}{dt} & = & r_\textrm{uncoat} V - r_\textrm{prep} U \\
\frac{dR}{dt} & = & r_\textrm{prep} U - \lambda_R R \\
\frac{dP}{dt} & = & r_\textrm{synth} R - \lambda_P P - r_\textrm{assemble} P \\
\frac{dA}{dt} & = & r_\textrm{assemble} P \underbrace{- E_{A,i} }_{\textrm{automatically done by PhysiCell}}
\end{eqnarray}

\subsubsection{Exocytosis} 
Next, we need to figure out net virus export. We use PhysiCell's standard form. If $\rho_A$ is the external concentration of exported (and assembled) viral particles, then: 
\begin{eqnarray}
\frac{d\rho_A}{dt} = ... \sum_{\textrm{cells } i} \delta( \vec{x}-\vec{x}_i ) E_{A,i}
\end{eqnarray}
where $\delta$ is the Dirac delta function centered at the cell's position
$\vec{x}_i$, and $E_{A,i}$ is the net export rate (in assembled virions per time). Integrating this, 
the rate change in exported assembled virions is

\begin{eqnarray}
\frac{dA_\textrm{external}}{dt} = ... \sum_{\textrm{cells } i} E_{A,i}.
\end{eqnarray}


We use the following constitutive relation: 
\begin{eqnarray}
E_A & = & r_\textrm{exo} A. 
\end{eqnarray}

As one final note: we want exported $A$ diffuse and be able to infect other cells. We do this with a ``cleanup'' step at each PhysiCell diffuison time step: in each voxel, we move all of $\rho_A$ to $\rho_V$ (the concentration 
of viral particles). 



\subsection{Receptor binding, endocytosis, and trafficking}
Define: \\

\begin{centering}
\begin{tabular}{rl}
$R_{EU}$ & external unbound ACE2 receptor \\
$R_{EB}$ & external bound ACE2 receptor (bound to virus) \\
$R_{IB}$ & internal bound ACE2 receptor (bound to virus, in an endosome) \\
$R_{IU}$ & internal unbound ACE2 receptor (released virus, on way back to membrane)
\end{tabular}
\end{centering}

External virus binds to $R_{EU}$ to create $R_{EB}$. We assume that only one virus can bind a single ACE2 receptor. 
$R_{EB}$ is trafficked inside to become $R_{IB}$. Virus is released from endosomes (when leaving $R_{IB}$) are a 
source term for $V$ above. 

Let's use this system of ODEs. If $n$ is the number of virions within interactoin distance of the 
cell's unbound receptors, 
\begin{eqnarray}
\frac{d R_{EU}}{dt} & = &  - r_\textrm{bind} n_V R_{EU} + \textrm{recycle} R_{IU} \\
\frac{d R_{EB}}{dt} & = &   r_\textrm{bind} n_V R_{EU} - r_\textrm{endo} R_{EB}  \\
\frac{d R_{IB}}{dt} & = &   r_\textrm{endo} R_{EB} - r_\textrm{release} R_{IB} \\
\frac{d R_{IU}}{dt} & = & r_\textrm{release} R_{IB} - r_\textrm{recycle} R_{IU} 
\end{eqnarray}

Matching the uptake term in the PDE to the binding term, we have 
\begin{equation}
r_\textrm{bind} n_V R_{EU} = U_i V_i \rho , 
\end{equation} 
and so 
\begin{eqnarray}
n_V & \approx & V_i \rho \\ 
U_i & = & r_\textrm{bind} R_{EU}. 
\end{eqnarray}
This gives us a constitutive relation where the uptake rate is modulated by available ACE2 receptors. 

Next, note that the release term gives the source rate of internalized virions for the virion model: 
\begin{equation}
\textrm{Source}_V  = r_\textrm{release} R_{IB} .
\end{equation}

\subsubsection{Implementation detail}
By setting the uptake rate $U_i = r_\textrm{bind} R_{EU}$, PhysiCell will automatically 
remove the correct amount of virions from the diffusing field and deliver them to 
internalized virions (in \verb|phenotype.molecular.internalized_total_substrates|). 

If this amount is cleared at each time step, then the amount in the internalized variable 
represents 
\begin{equation}
dR_{EB} = \Delta t r_\textrm{bind} n_V R_{EU} = \Delta t U_i V_i \rho.
\end{equation}
Thus, we call that $dR_{EB}$, deduct it from $R_{EU}$, add it to $R_{EB}$, and then 
set the ``internalized substrate'' quantity to zero. 

\subsubsection{Very preliminary ballpark parameter estimates}
\href{https://www.nature.com/articles/cr200815#MOESM255}{https://www.nature.com/articles/cr200815\#MOESM255} states about 3 hours to see endocytosed receptors, 13 hours (3 + 10 additional hours) to see few receptors in vessicles, and 14 hours to see recycled receptors. 

Based on this, let's say it takes 3 hours to move from $R_{EU}$ to $R_{EB}$ to $R_{IB}$, so that
\begin{equation}
% \frac{ R_{EU}(0) }{ r_\textrm{bind}} + \frac{1}{r_\textrm{endo}} = 180 \textrm{ min}. 
\frac{ 1 }{ r_\textrm{bind} R_{EU}(0) } + \frac{1}{r_\textrm{endo}} = 180 \textrm{ min}. 
\end{equation}
(We scale the first parameter by initial number of receptors to acount for the fact that this rate holds early when there are many free receptors.) 
Supposing that binding is relatively fast compared to endocytosis, let's suppose binding take on the order of 1 minute (for plentiful free receptors), and endocytosis requires on the order of 180 minutes. 

Continuing, we need 13-3 = 10 hours = 600 min to transit from $R_{IB}$ to $R_{IU}$, so $r_\textrm{release} = \frac{1}{600}$. 

Then, we need and additional 60 minutes to recycle the freed receptor (14 hours from start to finish), so $r_\textrm{recycle} = \frac{1}{60}.$ 

Since these are rough estimates, let's use only one significant digits: 
\begin{eqnarray}
R_{EU}(0) r_\textrm{bind} & = & 1 \textrm{ min}^{-1} \\ 
r_\textrm{endo} & = & \frac{1}{175} \sim 0.006 \textrm{ min}^{-1} \\ 
r_\textrm{release} & = & \frac{1}{600} \sim 0.002 \textrm{ min}^{-1} \\ 
r_\textrm{recycle} & = & \frac{1}{60} \sim 0.02 \textrm{ min}^{-1} \\ 
\end{eqnarray}
Assuming there are 1000 to 10000 receptors per cell, we will use a default 1000 receptors and set $r_\textrm{bind} = 0.001 $. Because the other estimates are rough, we wil round to nearest order of magnitude. Thus, 
\begin{eqnarray}
r_\textrm{bind} & = & 0.001 \textrm{ min}^{-1} \\ 
r_\textrm{endo} & = & 0.01 \textrm{ min}^{-1} \\ 
r_\textrm{release} & = & 0.001 \textrm{ min}^{-1} \\ 
r_\textrm{recycle} & = & 0.01 \textrm{ min}^{-1} \\ 
\end{eqnarray}

The paper states that viral proteins were expressed by about 18 hours. So, this will help constrain teh viral dynamics parameters.  Supposing it takes about 13 hours to release virion, the process of uncoating, preparing RNA, and synthesizing proteins takes on the order of 18-13 = 5 hours. Let's divide the time evening among these points, so that takes 100 minutes in each step. Then we will for now set 
\begin{eqnarray}
r_\textrm{uncoat} & = & 0.01 \textrm{ min}^{-1} \\ 
r_\textrm{prep} & = & 0.01 \textrm{ min}^{-1} \\ 
r_\textrm{synth} & = & 0.01 \textrm{ min}^{-1} 
\end{eqnarray}
Let us suppose for now that $r_\textrm{assemble} = r_\textrm{exo} = 0.01 \textrm{ min}^{-1}$. 
 


\subsection{Cell apoptotic response (PD)}
We use a fairly standard Hill function: let $e$ be the effect, which we model as increasing with assembled viral particles: 
\begin{eqnarray}
e & - & \frac{ A^n  }{ A^n + A_{\frac{1}{2}}^n  }
\end{eqnarray}
where $A_{\frac{1}{2}}$ is the half max, and $n$ is the Hill coefficient. 

Then, the cell's apoptosis rate is 
\begin{eqnarray}
r_{\textrm{apoptosis}} = \overline{r}_\textrm{max} e. 
\end{eqnarray}

At death, some fraction $f$ of the cell's $V$, $U$, $R$, $P$, and $A$ are released into the microenvironment. These can potentially be individually specified. 

\subsection{Later}
Use a molecular-scale model of ACE2 to model how the uptake rate changes. 

Use a receptor trafficking model to also modulate virion uptake. 

Better parameter estimates of viral replication.

Add immune response with more cell types. Cell death can release immunostimulatory factors that bring in macrophages. Those cause more damage and feedback. Connect that to ARDS. 









\section{v1 prototype}

\subsection{Diffusion}
We model external virus as a concentration that diffuses and can be uptake or exported. We use the standard PhysiCell equations: 

\begin{eqnarray}
\frac{\partial \rho }{\partial t} & = & 
D \nabla^2 \rho - \lambda \rho + 
\sum_{\textrm{cells }i} 
\delta( \vec{x} - \vec{x}_i ) \left[ 
-V_i U_i \rho + E_i 
\right]
\end{eqnarray}
where $D$ is the diffusion coefficient, $\lambda$ is the decay rate, $U$ is the cell's uptake rate, and $E$ is the net export rate (see more below). 

We use a PDE like this for viral particles $V$, uncoated virus $U$ (which might be released by dead cells), viral RNA $R$ (might be released by dead cells), viral protein $P$ (might be released by dead cells), and also 
release assembled virus (used as a technical assist on viral internalization and export dynamics--all virions in this last equation are immediately transferred to the viral concentration. See below). 

\subsection{Internal virus model}
Let's define the following variables: 
\begin{enumerate}
\item 
$V$ are viruses that are bound to the cell surface and in the process of endocytosis / adsorption. 

\item 
$U$ are viruses that have been fully endocytoses and coated. 

\item 
$R$ is RNA that has been delivered from uncoated virus $U$

\item 
$P$ are (full sets) of viral proteins that have been synthesized from RNA $R$

\item 
$A$ are assembled virions that are ready for export. 

\end{enumerate}
Let's model just the internal processes, noting that PhysiCell automatically takes care of virus intake (source for $V$) and export (sink for $A$). 

Our first model of this will be a system of ODEs: 
\begin{eqnarray}
\frac{dV}{dt} & = & -r_U V \\
\frac{dU}{dt} & = & r_U V - r_P U \\
\frac{dR}{dt} & = & r_P U - \lambda_R R \\
\frac{dP}{dt} & = & r_S R - \lambda_P P - r_A P \\
\frac{dA}{dt} & = & r_A P
\end{eqnarray}

Next, we need to figure out net virus export. We use PhysiCell's standard form. If $\rho_A$ is the external concentration of exported (and assembled) viral particles, then: 
\begin{eqnarray}
\frac{d\rho_A}{dt} = ... \sum_{\textrm{cells } i} \delta( \vec{x}-\vec{x}_i ) E_{A,i}
\end{eqnarray}
where $\delta$ is the Dirac delta function centered at the cell's position
$\vec{x}_i$, and $E_{A,i}$ is the net export rate (in assembled virions per time). 

We use the following constitutive relation: 
\begin{eqnarray}
E_A & = & r_E A. 
\end{eqnarray}

As one final note: we want exported $A$ diffuse and be able to infect other cells. We do this with a ``cleanup'' step at each PhysiCell diffuison time step: in each voxel, we move all of $\rho_A$ to $\rho_V$ (the concentration 
of viral particles). 

\subsection{Cell apoptotic response (PD)}
We use a fairly standard Hill function: let $e$ be the effect, which we model as increasing with assembled viral particles: 
\begin{eqnarray}
e & - & \frac{ A^n  }{ A^n + A_{\frac{1}{2}}^n  }
\end{eqnarray}
where $A_{\frac{1}{2}}$ is the half max, and $n$ is the Hill coefficient. 

Then, the cell's apoptosis rate is 
\begin{eqnarray}
r_{\textrm{apoptosis}} = \overline{r}_\textrm{max} e. 
\end{eqnarray}

At death, some fraction $f$ of the cell's $V$, $U$, $R$, $P$, and $A$ are released into the microenvironment. These can potentially be individually specified. 

\subsection{Later}
Use a molecular-scale model of ACE2 to model how the uptake rate changes. 

Use a receptor trafficking model to also modulate virion uptake. 

Better parameter estimates of viral replication.

Add immune response with more cell types. Cell death can release immunostimulatory factors that bring in macrophages. Those cause more damage and feedback. Connect that to ARDS. 


\end{document}